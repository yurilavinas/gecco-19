\section{Conclusion}
%Restating the aims of the study
The aim of the present research we investigated whether priority functions relate to MOEA/D.
We proposed two new priority functions for estimating difficulty and calculating priority in Resource Allocation, related to diversity. Therefore, we \emph{isolate} the priority functions in MOEA/D as the only variant to directly assess their efficiency. We defined a priority function based on the MRDL, which focus on diversity on the objective space. We also defined a priority based on the 2-Norm, which focus on diversity on the decision space. We then compared these new priority functions with the most popular approach, the Relative Improvement, or not using priority functions at all.

%Summarising main research findings
This study has shown that using 2-Norm of the difference of current solutions and its parents gave very good results in the artificial benchmark functions. Interestingly, these results were also superior that the results of the R.I.. In contrast, MRDL performed just slightly better than MOEA/D-DE. It could be that the reason for these results are that MRDL considers the diversity of a solution against all the population, while 2-Norm and R.I. consider only the a relationship of the current solution against its parent. In the other hand, the MRDL was the best priority function in the Lunar Landing Problem. Our findings suggest a role for diversity in promoting better performance in HV and IGD metrics as well as higher rates of non-dominated solutions.

%Suggesting implications for the field of knowledge
Overall, this study strengthens the idea that exploring priority function focusing on critical issues is worth of attention. This would be a fruitful area for further works. Also, we confirmed that R.I., a common priority function from the literature, can be a good choice depending on the MOP being addressed. 
%Suggesting implications for the field of knowledge and then Recognising the limitations of the current study
The findings of this investigation complement those of earlier studies that addressed the issue of Resource Allocation by using priority functions, such as MOEA/D-GRA\cite{zhou2016all}, by integrating diversity as priority function, and MOEA/D-CRA~\cite{kang2018collaborative}, by using only one population. We verified that indeed using only priority functions is effective to better allocate resources.

%Making recommendations for further research work and Setting out recommendations for practice or policy
Our findings suggest that in general priority functions should be considered as a simple yet efficient mechanism for improving the performance of MOEA/D as well as the rate of feasible and non-dominated solutions. However, we suggest that priority functions should be considered with care. We mean that greater efforts are needed to ensure which priority function is the most adequate for each problem and we recommend a more careful investigation of priority functions when designing MOEA for a specific goal.

. 

%The scope of this study is, therefore, limited to MOEA/Ds without archive populations. 
%
%This means that priority functions that concentrate resources to subproblems that generates more non-dominated solutions to the archive population were not considered in this study. Examples are the works of Cai and Lai~\cite{cai2015external} and Kang et al~\cite{kang2018collaborative}. In spite of its limitations, the study certainly adds to our understanding that addressing diversity in priority function is truly worth of attention.


%From the results on two artificial benchmark functions and one real-world problem, we understand that using priority functions related to diversity is one promising way of finding better results. More studies need to be conducted to understand how diversity really affects the improvement of performance. We also understand that in priority functions improve the performance of MOEA/D-DE in both HV and IGD metrics values as well as at the number of non-dominated solutions.% Finally, we add that using priority function could help to prioritize desired characteristics, as diversity (2-Norm) or convergence towards the PF (R.I.) and maybe some other features.

%There are many components and variants of MOEA/D and is interesting to combine 2-Norm and MRDL priority functions with the them. Then, we can better further explore the relationship of priority functions based on diversity with the others components and variants of the MOEA/D framework. How to define more efficient and effective utility functions for different problems is also worth further investigation (such as priority function that consider constraints) as well as verify the results of using priority function in others real-world problems.
%
