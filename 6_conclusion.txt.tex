\section{Conclusion}

In this paper, we have proposed two new priority functions, related to diversity. One based on the MRDL focus on diversity on the objective space while the 2-Norm focus on diversity on the decision space. We then compared the results with the priority function from the MOEA/D-DRA, R.I. and the MOEA/D-DE variant. 

To summarize the results, using a priority functions that is based on diversity, as the 2-Norm gave very good results, even better than the R.I. a common priority function from the literature. Therefore, we suggest exploring priority function based on diversity is a direction that should be further explore, for example in real-world problems. We expected that MRDL would give the best results in terms of improvement in performance, but to our surprise it barely helped. The reason might be that this method considers the diversity of a solution against all the population while the two best priority functions consider only the a relationship of the current solution against its parent. More effort should be directed to address the question of why random as priority function performed so well.

From the results on two artificial benchmark functions and one real-world problem, we understand that using priority functions related to diversity is one promising way of finding better results. However since only the results of the 2-Norm are solid, we more studies need to be conducted to understand how diversity really affects the improvement of performance. We also understand that in priority functions improve the performance of MOEA/D-DE in both HV and IGD metrics values as well as at the number of non-dominated solutions. Finally, we add that using priority function could help to prioritize desired characteristics, as diversity (2-Norm) or convergence towards the PF (R.I.) and maybe some other features.

There are many other components and variants of MOEA/D and is interesting to combine them with the 2 norm priority function to them. Then we can better understand the relationship of priority functions based on diversity with the others components and variants of the MOEA/D framework. How to define more efficient and effective utility functions for different problems is also worth further investigation as well as verify the results of using priority function in others real-world problems.

%It is in our understand that using priority functions in both real-world and artificial benchmarks improve the results of MOEA/D-DE. For all group of functions the performance of MOEA/D-DE, in terms of HV or IGD median values, was always suppressed by at least one variation using priority functions. 
