\section{Conclusion}
%Restating the aims of the study
The aim of the present research was to determine whether how priority functions relate to MOEA/D-DE. Therefore, we proposed we have proposed two new priority functions, related to diversity. One based on the MRDL focus on diversity on the objective space while the Norm focus on diversity on the decision space. We then compared these new priority functions with the priority function from the MOEA/D-DRA, R.I. against the MOEA/D-DE. 

%Summarising main research findings
This study has shown that using 2-Norm of the difference of current solutions and its parents gave very good results in the artificial benchmark functions. Interestingly, these results were little superior that the results of the R.I.. In contrast, MRDL performed just slightly better than MOEA/D-DE. These results were a little disappointing. It could be that this method considers the diversity of a solution against all the population while the two best priority functions consider only the a relationship of the current solution against its parent. In the other hand, the MRDL was the best priority function in the Lunar Landing Problem. The most surprising results was the one from Random, that was the best priority function in few functions. It is unfortunate that the study did not include a further understanding of the results of Random. Our findings suggest a role for diversity in promoting better performance in HV and IGD metrics as well as higher rates of non-dominated solutions 

%Suggesting implications for the field of knowledge
Overall, this study strengthens the idea that exploring priority function focusing on critical issues is worth of attention. This would be a fruitful area for further work. Also, we confirmed that R.I., a common priority function from the literature, can be a good choice depending on the MOP being addressed. 

%Making recommendations for further research work
These findings suggest that in general priority functions should be considered as a simple yet efficient mechanism for improving the performance of MOEA/D-DE and the rate of feasible and non-dominated solutions. 

%Setting out recommendations for practice or policy
Greater efforts are needed to ensure which priority function is the most adequate for each problem and we recommend a more careful analysis when designing MOEA with priority functions. 
%From the results on two artificial benchmark functions and one real-world problem, we understand that using priority functions related to diversity is one promising way of finding better results. More studies need to be conducted to understand how diversity really affects the improvement of performance. We also understand that in priority functions improve the performance of MOEA/D-DE in both HV and IGD metrics values as well as at the number of non-dominated solutions.% Finally, we add that using priority function could help to prioritize desired characteristics, as diversity (2-Norm) or convergence towards the PF (R.I.) and maybe some other features.

%There are many components and variants of MOEA/D and is interesting to combine 2-Norm and MRDL priority functions with the them. Then, we can better further explore the relationship of priority functions based on diversity with the others components and variants of the MOEA/D framework. How to define more efficient and effective utility functions for different problems is also worth further investigation (such as priority function that consider constraints) as well as verify the results of using priority function in others real-world problems.
%
