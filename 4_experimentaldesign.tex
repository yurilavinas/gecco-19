\section{Experimental Analysis}

%\begin{table}[!t]
%	\centering
%\begin{tabular}{@{}|l|l|@{}}
%		\toprule
%		\textbf{Parameters}   & \textbf{Values}          \\ \midrule
%		Initial value $u$     & 0.5, for every subproblem \\
%		Population size       & 150                      \\
%		Neighborhood size T & 20 \\
%		$\delta_p$ & 0.9 \\
%		$\phi$ & 0.5 \\
%		$\eta_m$ & 20 \\
%		$p_m$ & 0.03333333 \\
%		$n_r$ & 2 \\
%		\midrule
%		Number of evaluations & 60000 		\\
%		Number of repetitions & 21                  \\ \bottomrule
%
%	\end{tabular}
%\vspace{1em}
%\caption{Parameter settings.}
%\label{table1}
%\end{table}

To examine the effects of Resource Allocation under different priority functions on the MOEA/D, we perform a comparative experiment on benchmark functions and an optimization problem based on real world data.
In this experiment, we use the MOEA/D-DE implemented by the MOEADr package~\cite{moeadr_package}, modified to include Resource Allocation as described in the previous section. We compare five different RA strategies: No Resource Allocation, and RA using the following priority functions: MRDL, Norm, Relative Improvement and Random. In the following figures and tables, these strategies are referred, respectively, as: None, MRDL, Norm, R.I. and Random.

\subsection{Target Problems}

Two benchmark problem sets are used. The first one is the DTLZ function set~\cite{deb2005scalable}, with 100 dimensions and $k =$ dimensions - number of objectives $+ 1$, where the number of objectives is $2$. The second one is the UF function set~\cite{zhang2008multiobjective}, with 100 dimensions.

The Lunar Landing problem is an optimization problem about the selection of landing sites~\citep{MoonOrbitingSatellite2015}. In lunar exploration it is critical to find suitable landing sites for the rovers. A good landing site must provide enough sunshine for power supply, availability of nearby scientifically interesting materials, little communication interference, and low terrain inclination, among other considerations. The optimization problem is characterized by two decision variables: longitude and latitude; three minimization objectives: total continuous shade days, length of communication window (inverted), and inclination angle; and two constraints: maximum continuous shade days and maximum inclination. This problem is considered to be severely constrained, due to the presence of two craters in the landing area.

\subsection{Experimental Parameters}

We use the conventional MOEA/D-DE parameters~\cite{li2009multiobjective} for each Resource Allocation strategy: update size $nr = 2$, neighborhood size $T = 20$, and the neighborhood search probability $\delta_p = 0.9$. The DE mutation operator value is $phi=0.5$. The Polynomial mutation operator values are $\eta_m 20$ and $p_m = 0.03333333$. The decomposition function is Simple-Lattice Design (SLD), the scalar aggregation function is Weighted Sum (WS), the update strategy is the Restricted Update Strategy and we performed a simple linear scaling of the objectives to [0, 1].

For every strategy/function pair we perform 21 repetitions with 70000 function evaluations and population size $N=350$. Because the Lunar Landing problem is severely constrained, we used a much higher population size $N=5050$, and a slighly lower number of function evaluations (60000), following the winner of a recent competition using this problem~\cite{Competition2018}.

\subsection{Experimental Evaluation}

We compare the results of the different strategies based on their Hypervolume (HV) and Inverted Generational Distance~\footnote{IGD could not be calculated for the Lunar Landing problem, which has no Ideal Reference Pareto Front.} (IGD) metrics. Higher values of the HV indicate better approximations of the Pareto Front, while lower values of the IGD indicate better approximations. We also evaluate the proportion of non-dominated solutions and the number of feasible solutions.

For the calculation of HV, the objective function was scaled to the $0,1$ interval, and the reference point was set to $(1,1)$ for the 2-objective benchmark problems, $(1,1,1)$ for the 3-objective benchmark problems, and $(1,0,1)$ for the Lunar Landing Problem~\cite{Competition2018}).

To verify any statistical differences in the results for the different strategies, we use the Pairwise Wilcoxon Rank Sum Tests with confidence interval $\alpha = 0.05$ and with the Hommel adjustment method for multiple comparisons. For reproducibility purposes, all the code and data used in these experiments are available at [ANONYMIZED].
