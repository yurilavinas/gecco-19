\section{Results}


\begin{figure*}[!t]
	%	\Large{Average performance on different tournament size - Gallagher's Gaussian 21-hi Peaks Function}
	\begin{subfigure}[b]{0.33\textwidth}
		\centering
	\includegraphics[width=1\textwidth, height=1\textwidth]{images/UF3_HV}
%	\caption{HV - UF3}
	\end{subfigure}
	\begin{subfigure}[b]{0.33\textwidth}
		\centering
	\includegraphics[width=1\textwidth, height=1\textwidth]{images/UF8_HV}
%	\caption{HV - UF8}
	\end{subfigure}
\begin{subfigure}[b]{0.33\textwidth}
	\centering
	\includegraphics[width=1\textwidth, height=1\textwidth]{images/DTLZ4_HV}
%	\caption{HV - DTLZ4}
\end{subfigure}
	\caption{HV values of the last iteraction on Artificial Benchmark Problems}
		\label{HVS}
\end{figure*}

\begin{figure*}[!t]
%	\Large{Average performance on different tournament size - Gallagher's Gaussian 21-hi Peaks Function}
	\begin{subfigure}[b]{0.33\textwidth}
		\centering
		\includegraphics[width=1\textwidth, height=1\textwidth]{images/UF3hv_all}
	\caption{Evolution - UF3}
	\end{subfigure}
	\begin{subfigure}[b]{0.33\textwidth}
		\centering
		\includegraphics[width=1\textwidth, height=1\textwidth]{images/UF8hv_all}
	\caption{Evolution - UF8}
	\end{subfigure}
\begin{subfigure}[b]{0.33\textwidth}
	\centering
	\includegraphics[width=1\textwidth, height=1\textwidth]{images/DTLZ4hv_all}
	\caption{Evolution - DTLZ4}
\end{subfigure}
	\caption{Evolution of HV values on Artificial Benchmark Problems}
\label{evolution_hv}
\end{figure*}

\begin{figure*}[!t]

	\begin{subfigure}[b]{0.33\textwidth}
		\centering
		\includegraphics[width=1\textwidth, height=1\textwidth]{images/UF3_IGD}
%	\caption{IGD - UF3}
	\end{subfigure}
	\begin{subfigure}[b]{0.33\textwidth}
		\centering
		\includegraphics[width=1\textwidth, height=1\textwidth]{images/UF8_IGD}
%	\caption{IGD - UF8}
	\end{subfigure}
\begin{subfigure}[b]{0.33\textwidth}
	\centering
	\includegraphics[width=1\textwidth, height=1\textwidth]{images/DTLZ4_IGD}
%	\caption{IGD - DTLZ4}
\end{subfigure}
	\caption{IGD values of the last iteraction on Artificial Benchmark Problems}
		\label{IGDS}
\end{figure*}



\begin{figure*}[!t]

		\begin{subfigure}[b]{0.33\textwidth}
			\centering
		\includegraphics[width=1\textwidth, height=1\textwidth]{images/UF3igd_all}
	\caption{Evolution - UF3}
		\end{subfigure}
		\begin{subfigure}[b]{0.33\textwidth}
			\centering
		\includegraphics[width=1\textwidth, height=1\textwidth]{images/UF8igd_all}
	\caption{Evolution - UF8}
		\end{subfigure}
	\begin{subfigure}[b]{0.33\textwidth}
		\centering
		\includegraphics[width=1\textwidth, height=1\textwidth]{images/DTLZ4igd_all}
	\caption{Evolution - DTLZ4}
	\end{subfigure}
	\caption{Evolution of HV values on Artificial Benchmark Problems}
			\label{evolution_igd}
	\end{figure*}


\begin{figure*}[!t]
	
	%	\Large{Average performance on different tournament size - Gallagher's Gaussian 21-hi Peaks Function}
	\begin{subfigure}[b]{0.33\textwidth}
		\centering
		\includegraphics[width=1\textwidth, height=1\textwidth]{images/Ra-norm-uf3}
%		\caption{RA- UF8}
	\end{subfigure}
	\begin{subfigure}[b]{0.33\textwidth}
		\centering
		\includegraphics[width=1\textwidth, height=1\textwidth]{images/Ra-gra-uf3}
%		\caption{RA- UF8}
	\end{subfigure}
	\begin{subfigure}[b]{0.33\textwidth}
		\centering
		\includegraphics[width=1\textwidth, height=1\textwidth]{images/Ra-mrdl-uf3}
%		\caption{RA- UF8}
	\end{subfigure}
	\begin{subfigure}[b]{0.33\textwidth}
		\centering
		\includegraphics[width=1\textwidth, height=1\textwidth]{images/Ra-norm-uf8}
%		\caption{RA- UF8}
	\end{subfigure}
	\begin{subfigure}[b]{0.33\textwidth}
		\centering
		\includegraphics[width=1\textwidth, height=1\textwidth]{images/Ra-gra-uf8}
%		\caption{RA- UF8}
	\end{subfigure}
	\begin{subfigure}[b]{0.33\textwidth}
		\centering
		\includegraphics[width=1\textwidth, height=1\textwidth]{images/Ra-mrdl-uf8}
%		\caption{RA- UF8}
	\end{subfigure}
	\caption{Resource allocation by subproblem on Artificial Benchmark Problems}
	\label{RAs}
	
\end{figure*}

\begin{figure*}[!t]
	
	%	\Large{Average performance on different tournament size - Gallagher's Gaussian 21-hi Peaks Function}
	\begin{subfigure}[b]{0.33\textwidth}
		\centering
		\includegraphics[width=1\textwidth, height=1\textwidth]{images/moon_HV}
		\caption{HV values of the last iteraction on the Lunar Landing Problem}
	\end{subfigure}
	\begin{subfigure}[b]{0.33\textwidth}
		\centering
		\includegraphics[width=1\textwidth, height=1\textwidth]{images/moonhv_all}
		\caption{Evolution of the HV on the Lunar Landing}
	\end{subfigure}
	\caption{SBX crossover - ($\lambda, \lambda$) scheme.}
	\label{moon}
\end{figure*}


\begin{table*}[!t]
	\begin{tabular}{llllll}
		\cline{6-6}
		\hline
		\rowcolor[gray]{.7} \multicolumn{1}{|l|}{Priority Function:}         & \multicolumn{1}{l|}{None} & \multicolumn{1}{l|}{MRDL} & \multicolumn{1}{l|}{2-Norm} & \multicolumn{1}{l|}{R.I.} & \multicolumn{1}{l|}{Random} \\ \hline \hline  \hline
		\multicolumn{1}{|l|}{Lunar}           & \multicolumn{1}{l}{0.74 (0.10)} & \multicolumn{1}{l}{0.87 (0.11)} & \multicolumn{1}{l}{\textbf{0.93 (0.10)$*$}} & 0.76 (0.08)             &\multicolumn{1}{l|} {0.76 (0.08)} \\ \hline \hline
		\rowcolor[gray]{.95} \multicolumn{1}{|l|}{UF1}              & \multicolumn{1}{l}{0.861 (0.011)} & \multicolumn{1}{l}{0.863 (0.015)} & \multicolumn{1}{l}{0.833 (0.022)} & \textbf{0.877 (0.013)$*$}             &\multicolumn{1}{l|} {0.874 (0.015)$*$} \\ \hline \hline
		\multicolumn{1}{|l|}{UF2}              & \multicolumn{1}{l}{0.750 (0.009)} & \multicolumn{1}{l}{0.750 (0.005)} & \multicolumn{1}{l}{ 0.762 (0.010)$*$} & 0.772 (0.008)$*$             &\multicolumn{1}{l|} {\textbf{0.773 (0.008)$*$}} \\ \hline \hline
		\rowcolor[gray]{.95}\multicolumn{1}{|l|}{UF3}              & \multicolumn{1}{l}{0.844 (0.044)} & \multicolumn{1}{l}{0.860 (0.043)} & \multicolumn{1}{l}{ \textbf{0.944 (0.018)}$*$} & 0.918 (0.029)$*$             &\multicolumn{1}{l|}  {0.909 (0.037)$*$} \\ \hline \hline
		\multicolumn{1}{|l|}{UF4}              & \multicolumn{1}{l}{0.364 (0.005)} & \multicolumn{1}{l}{0.366 (0.003)} & \multicolumn{1}{l}{ 0.372 (0.003)}$*$ & 0.371 (0.004)$*$             &\multicolumn{1}{l|} {\textbf{0.373 (0.004)$*$}} \\ \hline \hline
		\rowcolor[gray]{.95}\multicolumn{1}{|l|}{UF5}              & \multicolumn{1}{l}{0.629 (0.022)} & \multicolumn{1}{l}{0.663 (0.024)$*$} & \multicolumn{1}{l}{ 0.754 (0.034)$*$} & \textbf{0.811 (0.015)$*$}             &\multicolumn{1}{l|} {0.810 (0.016)$*$} \\ \hline \hline
		\multicolumn{1}{|l|}{UF6}              & \multicolumn{1}{l}{0.661 (0.020)} & \multicolumn{1}{l}{0.660 (0.014)} & \multicolumn{1}{l}{ 0.662 (0.020)} & 0.686 (0.014)$*$             &\multicolumn{1}{l|} {\textbf{0.689 (0.015)$*$}} \\ \hline \hline
		\rowcolor[gray]{.95}\multicolumn{1}{|l|}{UF7}              & \multicolumn{1}{l}{0.803 (0.010)} & \multicolumn{1}{l}{0.801 (0.010)} & \multicolumn{1}{l}{ 0.818 (0.012)$*$} & \textbf{0.837 (0.005)$*$}             &\multicolumn{1}{l|} {0.834 (0.006)$*$} \\ \hline \hline
		\multicolumn{1}{|l|}{UF8}              & \multicolumn{1}{l}{0.894 (0.004)} & \multicolumn{1}{l}{0.900 (0.004)$*$} & \multicolumn{1}{l}{ 0.914 (0.005)$*$} & \textbf{0.922 (0.003)$*$}             &\multicolumn{1}{l|} {0.916 (0.004)$*$} \\ \hline \hline
		\rowcolor[gray]{.95}\multicolumn{1}{|l|}{UF9}              & \multicolumn{1}{l}{0.931 (0.004)} & \multicolumn{1}{l}{0.932 (0.004)} & \multicolumn{1}{l}{ \textbf{0.944 (0.014)$*$}} & 0.932 (0.004)             &\multicolumn{1}{l|} {0.940 (0.008)$*$} \\ \hline \hline
		\multicolumn{1}{|l|}{UF10}              & \multicolumn{1}{l}{0776 (0.017)} & \multicolumn{1}{l}{0.786 (0.017)} & \multicolumn{1}{l}{ 0.835 (0.035)$*$} & \textbf{0.861 (0.033)$*$}             &\multicolumn{1}{l|} {0.839 (0.026)$*$} \\ \hline  \hline
		\rowcolor[gray]{.95}\multicolumn{1}{|l|}{DTLZ1}              & \multicolumn{1}{l}{0.989 (0.003)} & \multicolumn{1}{l}{0.991 (0.004)$*$} & \multicolumn{1}{l}{ 0.997 (0.002)$*$} & \textbf{0.998 (0.002)$*$}             & \multicolumn{1}{l|} {\textbf{0.998 (0.001)$*$}} \\ \hline \hline
		\multicolumn{1}{|l|}{DTLZ2}              & \multicolumn{1}{l}{0.910 (0.002)} & \multicolumn{1}{l}{0.912 (0.002)$*$} & \multicolumn{1}{l}{ \textbf{0.922 (0.001)$*$}} & 0.921 (0.001)$*$             & \multicolumn{1}{l|} {\textbf{0.922 (0.001)$*$}} \\ \hline 
		\rowcolor[gray]{.95}\multicolumn{1}{|l|}{DTLZ3}              & \multicolumn{1}{l}{0.960 (0.015)} & \multicolumn{1}{l}{0.969 (0.016)$*$} & \multicolumn{1}{l}{ 0.992 (0.009)$*$} &0.991 (0.009)$*$             & \multicolumn{1}{l|} {\textbf{0.993 (0.006)$*$}} \\ \hline \hline
		\multicolumn{1}{|l|}{DTLZ4}              & \multicolumn{1}{l}{0.905 (0.003)} & \multicolumn{1}{l}{0.907 (0.004)$*$} & \multicolumn{1}{l}{ \textbf{0.920 (0.001)$*$}} & \textbf{0.920 (0.004)$*$}             & \multicolumn{1}{l|} {0.918 (0.002)$*$} \\ \hline 
		
	\end{tabular}
	\caption{Results of HV - Median and in parenthesis, the standard deviation. Highlighted in bold are the best values found by a priority function for that function. The priority function that has a Star $*$ is statistically different from MOEA/D-DE without priority function.}
	\label{table_hv}
\end{table*}

\begin{table*}[!t]
	\begin{tabular}{llllll}
		\cline{6-6}
		\hline
		\rowcolor[gray]{.7} \multicolumn{1}{|l|}{Priority Function:}         & \multicolumn{1}{l|}{None} & \multicolumn{1}{l|}{MRDL} & \multicolumn{1}{l|}{2-Norm} & \multicolumn{1}{l|}{R.I.} & \multicolumn{1}{l|}{Random} \\ \hline \hline
		\multicolumn{1}{|l|}{UF1}              & \multicolumn{1}{l}{0.140 (0.013)} & \multicolumn{1}{l}{0.128 (0.015)} & \multicolumn{1}{l}{0.109 (0.016)$*$} & \textbf{0.090 (0.012)$*$}             & \multicolumn{1}{l|} {0.093 (0.014)$*$} \\ \hline \hline
		\rowcolor[gray]{.95}\multicolumn{1}{|l|}{UF2}              & \multicolumn{1}{l}{0.082 (0.006)} & \multicolumn{1}{l}{0.080 (0.007)} & \multicolumn{1}{l}{ \textbf{0.060 (0.005)$*$}} & \textbf{0.060 (0.005)$*$}             & \multicolumn{1}{l|} {\textbf{0.060 (0.004)$*$}} \\ \hline \hline
		\multicolumn{1}{|l|}{UF3}              & \multicolumn{1}{l}{0.260 (0.012} & \multicolumn{1}{l}{0.257 (0.009)} 
		& \multicolumn{1}{l}{ \textbf{0.168 (0.025)}$*$} & 0.183 (0.335)$*$             &  \multicolumn{1}{l|} {0.214 (0.030)$*$} \\ \hline \hline
		\rowcolor[gray]{.95}\multicolumn{1}{|l|}{UF4}              & \multicolumn{1}{l}{0.100 (0.003)} & \multicolumn{1}{l}{ 0.100 (0.023)} & \multicolumn{1}{l}{ \textbf{0.095 (0.002)}$*$} & \textbf{0.095 (0.003)$*$}             & \multicolumn{1}{l|} {\textbf{0.095 (0.002)$*$}} \\ \hline \hline
		\multicolumn{1}{|l|}{UF5}              & \multicolumn{1}{l}{1.759 (0.080)} & \multicolumn{1}{l}{1.648 (0.091)$*$} & \multicolumn{1}{l}{ \textbf{0.972 (0.056)$*$}} & 1.056 (0.064)$*$            & \multicolumn{1}{l|} {1.085 (0.073)$*$} \\ \hline \hline
		\rowcolor[gray]{.95}\multicolumn{1}{|l|}{UF6}              & \multicolumn{1}{l}{0.121 (0.027)} & \multicolumn{1}{l}{0.120 (0.017)} & \multicolumn{1}{l}{ 0.100 (0.016)$*$} & \textbf{0.078 (0.014)$*$}            & \multicolumn{1}{l|} {0.079 (0.016)$*$} \\ \hline		 \hline
		\multicolumn{1}{|l|}{UF7}              & \multicolumn{1}{l}{0.125 (0.018)} & \multicolumn{1}{l}{0.127 (0.015)} & \multicolumn{1}{l}{ \textbf{0.061 (0.006)$*$}} & 0.068 (0.005)$*$             & \multicolumn{1}{l|} {0.074 (0.005)$*$} \\ \hline \hline
		\rowcolor[gray]{.95}\multicolumn{1}{|l|}{UF8}              & \multicolumn{1}{l}{0.286 (0.012)} & \multicolumn{1}{l}{0.279 (0.010)$*$} & \multicolumn{1}{l}{ \textbf{0.229 (0.014)$*$}} & 0.257 (0.020)$*$             & \multicolumn{1}{l|} {0.232 (0.006)$*$} \\ \hline \hline
		\multicolumn{1}{|l|}{UF9}              & \multicolumn{1}{l}{0.451 (0.012)} & \multicolumn{1}{l}{0.439 (0.015)$*$} & \multicolumn{1}{l}{ \textbf{0.385 (0.020)$*$}} & 0.420 (0.017)$*$             & \multicolumn{1}{l|} {0.400 (0.018)$*$} \\ \hline \hline
		\rowcolor[gray]{.95}\multicolumn{1}{|l|}{UF10}              & \multicolumn{1}{l}{3.693 (0.20)} & \multicolumn{1}{l}{3.456 (0.229)$*$} & \multicolumn{1}{l}{ 2.377 (0.241)$*$} & \textbf{2.364 (0.272)$*$}             & \multicolumn{1}{l|} {2.639 (0.253)$*$} \\ \hline  \hline  \hline
		\multicolumn{1}{|l|}{DTLZ1}              & \multicolumn{1}{l}{381.50 (125.13)} & \multicolumn{1}{l}{337.46 (164.94)} & \multicolumn{1}{l}{231.00 (086.40)$*$} & 222.46 (105.68)$*$             & \multicolumn{1}{l|} {\textbf{205.85 (093.83)$*$}} \\ \hline \hline
		\rowcolor[gray]{.95}\multicolumn{1}{|l|}{DTLZ2}              & \multicolumn{1}{l}{0.158 (0.013)} & \multicolumn{1}{l}{0.143 (0.010)$*$} & \multicolumn{1}{l}{ \textbf{0.072 (0.007)$*$}} & 0.095 (0.013)$*$             & \multicolumn{1}{l|} {0.085 (0.010)$*$} \\ \hline 
		\multicolumn{1}{|l|}{DTLZ3}              & \multicolumn{1}{l}{1248.4 (300.24)} & \multicolumn{1}{l}{1046.8 (405.65)} & \multicolumn{1}{l}{572.2 (312.88)$*$} & \textbf{495.2 (267.59)$*$}            & \multicolumn{1}{l|} {557.2 (234.31)$*$} \\ \hline \hline
		\rowcolor[gray]{.95}\multicolumn{1}{|l|}{DTLZ4}              & \multicolumn{1}{l}{0.1732 (0.024)} & \multicolumn{1}{l}{0.165 (0.037)$*$} & \multicolumn{1}{l}{ 0.076 (0.007)$*$} & \textbf{0.072 (0.077)$*$}             & \multicolumn{1}{l|} {0.093 (0.017)$*$} \\ \hline 
	\end{tabular}
	\caption{Results of IGD - Median and in parenthesis, the standard deviation. Highlighted in bold are the best values found by a priority function for that function. The priority function that has a Star $*$ is statistically different from MOEA/D-DE without priority function.}
	\label{table_igd}
\end{table*}

\begin{table*}[!t]
	\begin{tabular}{llllll}
		\cline{6-6}
		\hline
		\rowcolor[gray]{.7} \multicolumn{1}{|l|}{Priority Function:}         & \multicolumn{1}{l|}{None} & \multicolumn{1}{l|}{MRDL} & \multicolumn{1}{l|}{2-Norm} & \multicolumn{1}{l|}{R.I.} & \multicolumn{1}{l|}{Random} \\ \hline \hline
		\multicolumn{1}{|l|}{Lunar (Non-dominated (\%))}           & \multicolumn{1}{l}{0.69 (0.23)} & \multicolumn{1}{l}{0.87 (0.18)} & \multicolumn{1}{l}{0.83 (0.18)} & 0.69 (0.16)             &\multicolumn{1}{l|} {0.44 (0.16)} \\ \hline
		\multicolumn{1}{|l|}{Lunar (Time (seconds))}           & \multicolumn{1}{l}{32 (0.92)} & \multicolumn{1}{l}{39 (11.02)} & \multicolumn{1}{l}{46 (23.34)} & 2 (7.75)             &\multicolumn{1}{l|} {52 (0.86)} \\ \hline \hline
		\rowcolor[gray]{.95} \multicolumn{1}{|l|}{UF1 (Non-dominated (\%))}              & \multicolumn{1}{l}{0.28 (0.03)} & \multicolumn{1}{l}{0.29 (0.05)} & \multicolumn{1}{l}{0.94 (0.10)} & 0.47 (0.09)             &\multicolumn{1}{l|} {0.68 (0.06)} \\ \hline 
		\rowcolor[gray]{.95} \multicolumn{1}{|l|}{UF1 (Time (seconds))}              & \multicolumn{1}{l}{28 (2.36)} & \multicolumn{1}{l}{46 (2.16)} & \multicolumn{1}{l}{1 (0.60)} & 37 (3.05)            &\multicolumn{1}{l|} {34 (2.87)} \\ \hline \hline
		\multicolumn{1}{|l|}{UF2 (Non-dominated (\%))}           & \multicolumn{1}{l}{0.34 (0.04)} & \multicolumn{1}{l}{0.34 (0.05)} & \multicolumn{1}{l}{0.96 (0.05)} & 0.61 (0.16)             &\multicolumn{1}{l|} {0.81 (0.06)} \\ \hline 
		\multicolumn{1}{|l|}{UF2 (Time (seconds))}           & \multicolumn{1}{l}{24 (0)} & \multicolumn{1}{l}{41 (0.59)} & \multicolumn{1}{l}{1 (0.00)} & 34 (1.47)             &\multicolumn{1}{l|} {31 (0.00)} \\ \hline \hline
		\rowcolor[gray]{.95} \multicolumn{1}{|l|}{UF3 (Non-dominated (\%))}              & \multicolumn{1}{l}{0.22 (0.03)} & \multicolumn{1}{l}{0.24 (0.03)} & \multicolumn{1}{l}{0.78 (0.13)} & 0.44 (0.15)            &\multicolumn{1}{l|} {0.59 (0.07)} \\ \hline 
		\rowcolor[gray]{.95} \multicolumn{1}{|l|}{UF3 (Time (seconds))}              & \multicolumn{1}{l}{25 (0.22)} & \multicolumn{1}{l}{43 (1.55)} & \multicolumn{1}{l}{1 (0.60)} & 55 (2.09)           &\multicolumn{1}{l|} {31 (0.00)} \\ \hline \hline
		\multicolumn{1}{|l|}{UF4 (Non-dominated (\%))}           & \multicolumn{1}{l}{0.58 (0.05)} & \multicolumn{1}{l}{0.60 (0.04)} & \multicolumn{1}{l}{0.90 (0.08)} & 0.75 (0.06)             &\multicolumn{1}{l|} {0.82 (0.04)} \\ \hline
		\multicolumn{1}{|l|}{UF4 (Time (seconds))}           & \multicolumn{1}{l}{24 (0.00)} & \multicolumn{1}{l}{41 (0.00)} & \multicolumn{1}{l}{1 (0.00)} & 34 (1.56)             &\multicolumn{1}{l|} {31 (0.22)} \\ \hline \hline
		\rowcolor[gray]{.95} \multicolumn{1}{|l|}{UF5 (Non-dominated (\%))}              & \multicolumn{1}{l}{0.21 (0.03)} & \multicolumn{1}{l}{0.26 (0.04)} & \multicolumn{1}{l}{0.96 (0.03)} & 0.69 (0.13)             &\multicolumn{1}{l|} {0.71 (0.08)} \\ \hline
		\rowcolor[gray]{.95} \multicolumn{1}{|l|}{UF5 (Time (seconds))}              & \multicolumn{1}{l}{24 (0.22)} & \multicolumn{1}{l}{41 (0.60)} & \multicolumn{1}{l}{2 (0.44)} & 33 (0.54)             &\multicolumn{1}{l|} {31 (0.22)} \\ \hline \hline
		\multicolumn{1}{|l|}{UF6 (Non-dominated (\%))}           & \multicolumn{1}{l}{0.27 (0.03)} & \multicolumn{1}{l}{0.27 (0.03)} & \multicolumn{1}{l}{0.95 (0.08)} & 0.45 (0.07)             &\multicolumn{1}{l|} {0.61 (0.07)} \\ \hline 
		\multicolumn{1}{|l|}{UF6 (Time (seconds))}           & \multicolumn{1}{l}{24 (0.36)} & \multicolumn{1}{l}{41 (0.79)} & \multicolumn{1}{l}{1 (0.00)} & 33 (0.51)             &\multicolumn{1}{l|} {31 (0.00)} \\ \hline \hline
		\rowcolor[gray]{.95} \multicolumn{1}{|l|}{UF7 (Non-dominated (\%))}              & \multicolumn{1}{l}{0.38 (0.06)} & \multicolumn{1}{l}{0.37 (0.05)} & \multicolumn{1}{l}{0.96 (0.03)} & 0.57 (0.11)             &\multicolumn{1}{l|} {0.80 (0.04)} \\ \hline
		\rowcolor[gray]{.95} \multicolumn{1}{|l|}{UF7 (Time (seconds))}              & \multicolumn{1}{l}{24 (0.30)} & \multicolumn{1}{l}{42 (0.38)} & \multicolumn{1}{l}{1 (0.46)} & 34 (0.90)             &\multicolumn{1}{l|} {31 (0.30)} \\ \hline \hline
		\multicolumn{1}{|l|}{UF8 (Non-dominated (\%))}           & \multicolumn{1}{l}{0.40 (0.03)} & \multicolumn{1}{l}{0.39 (0.03)} & \multicolumn{1}{l}{0.70 (0.06)} & 0.67 (0.07)             &\multicolumn{1}{l|} {0.63 (0.03)} \\ \hline 
		\multicolumn{1}{|l|}{UF8 (Time (seconds))}           & \multicolumn{1}{l}{24 (0.00)} & \multicolumn{1}{l}{39 (0.58)} & \multicolumn{1}{l}{1 (0.00)} & 51 (2.23)             &\multicolumn{1}{l|} {31 (0.30)} \\ \hline \hline
		\rowcolor[gray]{.95} \multicolumn{1}{|l|}{UF9 (Non-dominated (\%))}              & \multicolumn{1}{l}{0.32 (0.03)} & \multicolumn{1}{l}{0.33 (0.03)} & \multicolumn{1}{l}{0.52 (0.03)} & 0.49 (0.04)             &\multicolumn{1}{l|} {0.48 (0.03)} \\ \hline
		\rowcolor[gray]{.95} \multicolumn{1}{|l|}{UF9 (Time (seconds))}              & \multicolumn{1}{l}{25 (0.50)} & \multicolumn{1}{l}{23 (4.59)} & \multicolumn{1}{l}{1 (27.84)} & 47 (2.75)             &\multicolumn{1}{l|} {32 (0.48)} \\ \hline \hline
		\multicolumn{1}{|l|}{UF10 (Non-dominated (\%))}           & \multicolumn{1}{l}{0.43 (0.04)} & \multicolumn{1}{l}{0.46 (0.05)} & \multicolumn{1}{l}{0.75 (0.05)} & 0.68 (0.08)             &\multicolumn{1}{l|} {0.73 (0.06)} \\ \hline 
		\multicolumn{1}{|l|}{UF10 (Time (seconds))}           & \multicolumn{1}{l}{25 (0.51)} & \multicolumn{1}{l}{38 (1.83)} & \multicolumn{1}{l}{1 (0.00)} & 47 (3.51)             &\multicolumn{1}{l|} {32 (0.30)} \\ \hline \hline
		\rowcolor[gray]{.95} \multicolumn{1}{|l|}{DTLZ1 (Non-dominated (\%))}              & \multicolumn{1}{l}{0.05 (0.01)} & \multicolumn{1}{l}{0.07 (0.01)} & \multicolumn{1}{l}{0.98 (0.06)} & 0.62 (0.17)             &\multicolumn{1}{l|} {0.68 (0.13)} \\ \hline
		\rowcolor[gray]{.95} \multicolumn{1}{|l|}{DTLZ1 (Time (seconds))}              & \multicolumn{1}{l}{25 (0.90)} & \multicolumn{1}{l}{41 (1.16)} & \multicolumn{1}{l}{2 (0.00)} & 36 (10.19)             &\multicolumn{1}{l|} {32 (0.00)} \\ \hline \hline
		\multicolumn{1}{|l|}{DTLZ2 (Non-dominated (\%))}           & \multicolumn{1}{l}{0.18 (0.03)} & \multicolumn{1}{l}{0.25 (0.03)} & \multicolumn{1}{l}{0.96 (0.05)} & 0.73 (2.93)             &\multicolumn{1}{l|} {0.84 (0.07)} \\ \hline 
		\multicolumn{1}{|l|}{DTLZ2 (Time (seconds))}           & \multicolumn{1}{l}{25 (0.30)} & \multicolumn{1}{l}{41 (0.54)} & \multicolumn{1}{l}{1 (12.66)} & 40 (2.93)             &\multicolumn{1}{l|} {32 (0.60)} \\ \hline \hline\rowcolor[gray]{.95} \multicolumn{1}{|l|}{DTLZ3 (Non-dominated (\%))}              & \multicolumn{1}{l}{0.04 (0.02)} & \multicolumn{1}{l}{0.04 (0.02)} & \multicolumn{1}{l}{0.99 (0.05)} & 0.32 (0.17)             &\multicolumn{1}{l|} {0.44 (0.18)} \\ \hline
		\rowcolor[gray]{.95} \multicolumn{1}{|l|}{DTLZ3 (Time (seconds))}              & \multicolumn{1}{l}{25 (0.22)} & \multicolumn{1}{l}{40 (0.78)} & \multicolumn{1}{l}{2 (0.51)} & 32 (3.09)             &\multicolumn{1}{l|} {32 (0.22)} \\ \hline \hline
		\multicolumn{1}{|l|}{DTLZ4 (Non-dominated (\%))}           & \multicolumn{1}{l}{0.11 (0.03)} & \multicolumn{1}{l}{0.13 (0.03)} & \multicolumn{1}{l}{0.96 (0.03)} & 0.7 (0.24)             &\multicolumn{1}{l|} {0.63 (0.13)} \\ \hline 
		\multicolumn{1}{|l|}{DTLZ4 (Time (seconds))}           & \multicolumn{1}{l}{26 (0.36)} & \multicolumn{1}{l}{43 (1.01)} & \multicolumn{1}{l}{1 (0.00)} & 47 (5.94)             &\multicolumn{1}{l|} {33 (0.68)} \\ \hline \hline
		
	\end{tabular}
	\caption{Percentage of the median values and the standard deviation in parenthesis of non-dominated solutions and of the execution time.}
	\label{minor_results}
\end{table*}

It is in our understand that using priority functions in both real-world and artificial benchmarks improve the results of MOEA/D-DE. For all group of functions the performance of MOEA/D-DE, in terms of HV or IGD median values, was always suppressed by at least one variation using priority functions. 

\subsection{UF benchmark functions}

Figure~\ref{HVS} shows box-plot that exemplify the results found in the UF benchmark functions in terms of the HV values while Figure~\ref{IGDS} does the same but in terms of the IGD values. In them we can see that MRDL as a priority function is slightly better than MOEA/D-DE, with the difference of the median being statistical different in the UF5 and UF8, considering both HV and IGD values. 2-Norm, R.I. and Random perform the best, in both terms of median and standard deviation and that 

Turning to the results of the Table~\ref{table_hv}, we discuss the results of every priority function. The 2-norm as priority function lead to several good results in median of the HV values while had very good results in the median of the IGD values. for HV values it had the best median in the UF3 and UF9 cases. Although for the UF3 problem, there is no statistical significant difference  for the R.I. while for the UF9, there is no statistical significant difference for the Random. For the UF4 and UF10 its median value is not the best, but this difference is not statistical significant. The relative improvement function was first introduced in the context of the unconstrained MOEA competition in the CEC 2009~\cite{zhang2009performance}, being the winner of that competition~\cite{zhang2008multiobjective}. This competition introduced the UF benchmark functions, so it came to us with no surprising the good results from the relative improvement priority function. In terms of HV values, in five functions it had the higher median. Only in UF8 results there exists a statistical significant difference to the other priority functions.  In UF1, UF5 and UF7 statistical significant difference to the results of the random. Surprisingly, the Random priority function got the higher median in the UF2 (with no statistical difference to R.I.), UF4(with no statistical difference to R.I. or 2-Norm) and UF6 functions(with no statistical difference to R.I.) with its median being always higher that the one from MRDL.

Moving to the IGD results, in the Table~\ref{table_igd} we verify that the 2-Norm had the best median in 7 functions, being statistically different to all others in the UF3, UF5, UF7 and UF9 functions. In the UF2 and UF4 the results of the 2-Norm are not statistically different from the R.I. or the Random. In UF8 the results of the 2-Norm not statistically different from the Random. The R.I. had the highest median in the UF1, UF2, UF4, UF6 and UF10 functions. As in the case of UF2, the results of UF1 and UF4 are not statistically different from the R.I. or the Random. In UF6 the results are not statistically different from the Random while on the UF10, there is no statistical different to the results of the 2-Norm.


 Figures~\ref{evolution_hv} and~\ref{evolution_igd} show how the values of the HV and IGD evolve over the iteractions of the algorithms. For the UF3 function the results caught our attention. First, it seems that more evaluations are need to a convergence of the values. Second, for the HV values the priority functions relative improvement and 2-norm made a "jump" over 30000 and 60000 evaluations, improving fast the HV metric values. Also at the end fo the evaluations, there is a strange regressive peak of the HV and IGD values of the 2-Norm. That said, in most cases the values of HV and IGD converged by the end of the number evaluations (70000).

\subsection{DTLZ benchmark functions}

Figure~\ref{HVS} shows box-plot that exemplify the results found in the DTLZ benchmark functions in terms of the HV values while Figure~\ref{IGDS} does the same but in terms of the IGD values. In them t we can see that MRDL as a priority function is slightly better than MOEA/D-DE in terms of median.  With R.I., 2-Norm and Random performing the best.


%Turning to the results of the Tables~\ref{table_hv} and ~\ref{table_igd}, we discuss the results of every priority function. The 2-norm as priority function lead to several good results in median of the HV values while had very good results in the median of the IGD values. We highlight in the UF-5 function, being clearly better than the others. The relative improvement function was first introduced in the context of the unconstrained MOEA competition in the CEC 2009~\cite{zhang2009performance}, being the winner of that competition~\cite{zhang2008multiobjective}. This competition introduced the UF benchmark functions, so it came to us with no surprising the good results from the relative improvement priority function. In terms of HV values, in all but in the UF3 and UF-9 benchmark function it had the best result, being significantly different from the MOEA/D-DE in every single case. The results of the relative improvement in terms of median values of the IGD values was not as remarkable, but they are still very good. For our great surprise the Random priority function performed quiet well (and better than the MRDL). 

Figures~\ref{evolution_hv} and~\ref{evolution_igd} show how the values of the HV and IGD evolve over the iteractions of the algorithms. In most cases the values of HV and IGD converged by the end of the number evaluations (70000).



\subsection{Lunar Landing Problem}


Figure~\ref{moon} (a) show box-plot the results found in the Lunar Landing benchmark problem in terms of the HV values. Both priority functions related to diversity, MRDL and 2-norm, improved a lot the performance of the MOEA/D-DE. On the other hand, the relative improvement lead to a diminish of the standard deviation, with just a slight improvement on the performance.  Figure~\ref{moon} (b) show how the values of the HV evolve over the iteractions of the algorithms. All of the variants had converged around 20000 evaluations, with the exception of the MRDL (it converged with 40000). All used around half or less of the total number evaluations (70000) to converge.

Changing to the results of the Table~\ref{table_hv}, we discuss the results of medians and standard deviation the priority functions.  We highlight the outstanding result of 2-norm since it was clearly better than the other priority function results in terms of median of HV values as it is statistically different from MOEA/D-DE, yet its standard deviation was very high. The results of MRDL were not as good as the previous one, however, it is still had impressive median HV values results, but without statistically significant difference from MOEA/D-DE. The relative improvement improved the results of MOEA/D-DE but not as much as in the case of the UF benchmark functions, where it had frequently the best results. Here, on the other hand, the main improvement was in terms of diminishing the standard deviation. More results can be found at the supplementary materials.



\subsection{Resource Allocation}

Figure~\ref{RAs} illustrates the amount resource allocated by Norm, R.I. and MRDL to every subproblem on UF 3 and 8 problems. We exclude the visuals from MOEAD/D-DE since it give the same amount of resource to every problem (200) and of the Random, since it is completely noise. 

It is clear from Figures~\ref{RAs} that, during the execution of the algorithm, the resource allocate to each subproblem is different. This behavior is different given priority functions, illustrating that every priority function allocates different amount of resource given their characteristics. It called our attention the results form the priority functions Norm and R.I., since it appears to be that they prioritized subproblems in an opposite way. It is also important to highlight that each priority function influences the search differently given the context of the MOP. At first, it is not clear how MRDL influences the resource allocation in a general way, however when looking at the results of MRDL, we cogitate that if a solution is given more resources its neighbor solutions and vice-versa


\subsection{Non-Dominated Solutions and Execution Time}

The results on the Table~\ref{minor_results} indicate that the 2-Norm leads to the best rate of non-dominated solutions in any benchmark function.  Not only that but it is always the fastest. The MRDL priority function improved a little the rate of non-dominated solutions, at the cost of a longer execution time, which may be alleviated by more costly problems, such as real-world problems.  The same behavior is found in the R.I. and Random results. More results can be found at the supplementary materials.




