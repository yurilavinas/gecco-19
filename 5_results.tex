\section{Results}

\begin{table*}[!t]
	\begin{tabular}{llllll}
		\cline{6-6}
		\hline
		\multicolumn{1}{|l|}{PF:}         & \multicolumn{1}{l|}{None} & \multicolumn{1}{l|}{MRDL} & \multicolumn{1}{l|}{Norm} & \multicolumn{1}{l|}{R.I.} & \multicolumn{1}{l|}{Random} \\ \hline \hline
		\multicolumn{1}{|l|}{Lunar}           & \multicolumn{1}{l}{0.74 (0.10)} & \multicolumn{1}{l}{0.87 (0.11)} & \multicolumn{1}{l}{\textbf{0.93} (0.10)$*$} & 0.76 (0.08)             &\multicolumn{1}{l|} {0.76 (0.08)} \\ \hline \hline
		\multicolumn{1}{|l|}{UF1}              & \multicolumn{1}{l}{0.86 (0.02)} & \multicolumn{1}{l}{0.86 (0.01)} & \multicolumn{1}{l}{0.83 (0.02)} & \textbf{0.88 (0.01)$*$}             & \textbf{0.88 (0.01)$*$} \\ \hline
		\multicolumn{1}{|l|}{UF2}              & \multicolumn{1}{l}{0.75 (0.01)} & \multicolumn{1}{l}{0.75 (0.01)} & \multicolumn{1}{l}{ 0.76 (0.01)$*$} & \textbf{0.77 (0.08)$*$}             & \textbf{0.77 (0.01)$*$} \\ \hline
		\multicolumn{1}{|l|}{UF3}              & \multicolumn{1}{l}{0.84 (0.04)} & \multicolumn{1}{l}{0.86 (0.04)} & \multicolumn{1}{l}{ \textbf{0.94 (0.02)}$*$} & 0.92 (0.03)$*$             &  0.91 (0.04)$*$ \\ \hline
		\multicolumn{1}{|l|}{UF4}              & \multicolumn{1}{l}{0.36 (0.01)} & \multicolumn{1}{l}{\textbf{0.37 (0.01)}} & \multicolumn{1}{l}{ \textbf{0.37 (0.01)}$*$} & \textbf{0.37 (0.01)$*$}             & \textbf{0.37 (0.01)$*$} \\ \hline
		\multicolumn{1}{|l|}{UF5}              & \multicolumn{1}{l}{0.63 (0.02)} & \multicolumn{1}{l}{0.66 (0.02)$*$} & \multicolumn{1}{l}{ 0.75 (0.03)$*$} & \textbf{0.81 (0.01)$*$}             & \textbf{0.81 (0.02)$*$} \\ \hline
		\multicolumn{1}{|l|}{UF6}              & \multicolumn{1}{l}{0.66 (0.02)} & \multicolumn{1}{l}{0.66 (0.01)} & \multicolumn{1}{l}{ 0.66 (0.02)} & \textbf{0.69 (0.01)$*$}             & \textbf{0.69 (0.01)$*$} \\ \hline
		
		\multicolumn{1}{|l|}{UF7}              & \multicolumn{1}{l}{0.80 (0.01)} & \multicolumn{1}{l}{0.80 (0.01)} & \multicolumn{1}{l}{ 0.82 (0.01)$*$} & \textbf{0.84 (0.01)$*$}             & 0.83 (0.01)$*$ \\ \hline
		UF8                                           &                          &                          &                          &                           &                             \\
		UF9                                           &                          &                          &                          &                           &                             \\
		UF10                                          &                          &                          &                          &                           &                            
	\end{tabular}
	\caption{HV Results - Lunar problem: The best algorithm is the one that uses Norm. It is the only one that is stats different from None, although by median all priority functions are better. Highlighted are the best values found.Star $*$ means stat diff from MOEA/D-DE without priority function. SD Values smaller than 0.01 were truncated to that\\ for UF the results are}
	\label{table_hv}
\end{table*}

\begin{table*}[!t]
	\begin{tabular}{llllll}
		\cline{6-6}
		\hline
		\multicolumn{1}{|l|}{PF:}         & \multicolumn{1}{l|}{None} & \multicolumn{1}{l|}{MRDL} & \multicolumn{1}{l|}{Norm} & \multicolumn{1}{l|}{R.I.} & \multicolumn{1}{l|}{Random} \\ \hline \hline
		\multicolumn{1}{|l|}{UF1}              & \multicolumn{1}{l}{0.14 (0.01)} & \multicolumn{1}{l}{0.13 (0.01)} & \multicolumn{1}{l}{0.10 (0.02)$*$} & \textbf{0.09 (0.01)$*$}             & \textbf{0.09 (0.01)$*$} \\ \hline
		\multicolumn{1}{|l|}{UF2}              & \multicolumn{1}{l}{0.08 (0.01)} & \multicolumn{1}{l}{0.08 (0.01)} & \multicolumn{1}{l}{ \textbf{0.06 (0.01)$*$}} & \textbf{0.06 (0.01)$*$}             & \textbf{0.06 (0.01)$*$} \\ \hline
		\multicolumn{1}{|l|}{UF3}              & \multicolumn{1}{l}{0.26 (0.01} & \multicolumn{1}{l}{0.26 (0.01)} & \multicolumn{1}{l}{ \textbf{0.17 (0.02)}$*$} & 0.18 (0.03)$*$             &  0.21 (0.03)$*$ \\ \hline
		\multicolumn{1}{|l|}{UF4}              & \multicolumn{1}{l}{0.10 (0.01)} & \multicolumn{1}{l}{ 0.10 (0.01)} & \multicolumn{1}{l}{ \textbf{0.09 (0.01)}$*$} & \textbf{0.09 (0.01)$*$}             & \textbf{0.09 (0.01)$*$} \\ \hline
		\multicolumn{1}{|l|}{UF5}              & \multicolumn{1}{l}{1.75 (0.08)} & \multicolumn{1}{l}{1.65 (0.09)$*$} & \multicolumn{1}{l}{ \textbf{0.97 (0.06)$*$}} & 1.05 (0.06)$*$            & 1.08(0.07)$*$ \\ \hline
		\multicolumn{1}{|l|}{UF6}              & \multicolumn{1}{l}{0.12 (0.03)} & \multicolumn{1}{l}{0.12 (0.02)} & \multicolumn{1}{l}{ 0.10 (0.07)$*$} & \textbf{0.08 (0.01)$*$}            & \textbf{0.08 (0.02)$*$} \\ \hline		
		\multicolumn{1}{|l|}{UF7}              & \multicolumn{1}{l}{0.12 (0.02)} & \multicolumn{1}{l}{0.13 (0.01)} & \multicolumn{1}{l}{ \textbf{0.06 (0.01)$*$}} & 0.07 (0.01)$*$             & 0.07 (0.01)$*$ \\ \hline
		UF8                                           &                          &                          &                          &                           &                             \\
		UF9                                           &                          &                          &                          &                           &                             \\
		UF10                                          &                          &                          &                          &                           &                            
	\end{tabular}
	\caption{Highlighted are the best values found.Star $*$ means stat diff from MOEA/D-DE without priority function. SD Values smaller than 0.01 were truncated to that\\ for UF the results are}
		\label{table_igd}
\end{table*}


\begin{figure*}[!t]

	%	\Large{Average performance on different tournament size - Gallagher's Gaussian 21-hi Peaks Function}
	\begin{subfigure}[b]{0.33\textwidth}
		\centering
		\includegraphics[width=1\textwidth, height=1\textwidth]{images/moon_HV}
		\caption{HV values of the last iteraction on the Lunar Landing Problem}
	\end{subfigure}
	\begin{subfigure}[b]{0.33\textwidth}
		\centering
	\includegraphics[width=1\textwidth, height=1\textwidth]{images/UF3_HV}
	\caption{HV values of the last iteraction on the UF-3 function Problem}
	\end{subfigure}
	\begin{subfigure}[b]{0.33\textwidth}
		\centering
	\includegraphics[width=1\textwidth, height=1\textwidth]{images/UF8_HV}
	\caption{HV values of the last iteraction on the UF-8 function Problem}
	\end{subfigure}
	\caption{SBX crossover - ($\lambda, \lambda$) scheme.}
		\label{HVS}
\end{figure*}

\begin{figure*}[!t]
%	\Large{Average performance on different tournament size - Gallagher's Gaussian 21-hi Peaks Function}
	\begin{subfigure}[b]{0.33\textwidth}
		\centering
		\includegraphics[width=1\textwidth, height=1\textwidth]{images/moonhv_all}
		\caption{Evolution of the HV on the Lunar Landing}
	\end{subfigure}
	\begin{subfigure}[b]{0.33\textwidth}
		\centering
		\includegraphics[width=1\textwidth, height=1\textwidth]{images/UF3hv_all}
		\caption{Evolution of the HV on the UF3}
	\end{subfigure}
	\begin{subfigure}[b]{0.33\textwidth}
		\centering
		\includegraphics[width=1\textwidth, height=1\textwidth]{images/UF8hv_all}
		\caption{Evolution of the HV on the UF8}
	\end{subfigure}
	\caption{SBX crossover - ($\lambda, \lambda$) scheme.}
\label{evolution_hv}
\end{figure*}

\begin{figure*}[!t]

	\begin{subfigure}[b]{0.49\textwidth}
		\centering
		\includegraphics[width=1\textwidth, height=0.7\textwidth]{images/UF3_IGD}
		\caption{IGD values of the last iteraction on the UF-3 function}
	\end{subfigure}
	\begin{subfigure}[b]{0.49\textwidth}
		\centering
		\includegraphics[width=1\textwidth, height=0.7\textwidth]{images/UF8_IGD}
		\caption{IGD values of the last iteraction on the UF-3 function}
	\end{subfigure}
	\caption{SBX crossover - ($\lambda, \lambda$) scheme.}
		\label{IGDS}
\end{figure*}



\begin{figure*}[!t]

		\begin{subfigure}[b]{0.49\textwidth}
			\centering
		\includegraphics[width=1\textwidth, height=0.7\textwidth]{images/UF3igd_all}
			\caption{Evolution of the IGD on the UF3}
		\end{subfigure}
		\begin{subfigure}[b]{0.49\textwidth}
			\centering
		\includegraphics[width=1\textwidth, height=0.7\textwidth]{images/UF8igd_all}
			\caption{Evolution of the IGD on the UF8}
		\end{subfigure}
		\caption{SBX crossover - ($\lambda, \lambda$) scheme.}
			\label{evolution_igd}
	\end{figure*}


First we give the overall results, and then we discuss in more specific details given the group pf benchmark problems.
\subsection{Overall results}



It is in our understand that using priority functions in both real-world and artificial benchmarks improve the results of MOEA/D-DE. For all group of functions the performance of MOEA/D-DE, in terms of HV or IGD median values, was always suppressed by at least one variation using priority functions. 

For all groups of functions, "spectral" norm had the best results given the median of HV or IGD, with except for the  UF-1  and Uf-6 functions. This priority function had outstanding results in the Lunar Landing Problem. MRDL as a priority function had not improved much the performance of MOEA/D-DE, in terms of median values of HV or IGD, by the clear exception of the Lunar Landing Problem, although with no statistical difference. The results of the MRDL priority function only lead to a statistical improvement in one function, UF-5, for both metrics values while in the UF-4, for the HV value, it showed the best value (the same as the others priority functions). Relative improvement performed the best when comparing with MOEA/D-DE with statistical difference from MOEA/D-DE in all but the Lunar Landing problem. In most cases, given both metrics values, it had the best median results. 



\subsection{UF benchmark functions}

Figure~\ref{HVS} (b) and (c) show box-plot that exemplify the results found in the UF benchmark functions in terms of the HV values while Figure~\ref{IGDS} (a) and (b) does the same but in terms of the IGD values. In it we can see that MRDL as a priority function is slightly better than MOEA/D-DE in terms of median and standard deviation.  Figures~\ref{evolution_hv} (b) and (c) and~\ref{evolution_igd} (a) and (b) show how the values of the HV and IGD evolve over the iteractions of the algorithms. For the UF-3 function the results caught our attention. First, it seems that more evaluations are need to a convergence of the values. Second, for the HV values the priority functions relative improvement and "spectral" norm made a "jump" over 30000 and 60000 evaluations, improving fast the HV metric values. Also at the end fo the evaluations, there is a strange regressive peak of the HV and IGD values of the "spectral norm".

Turning to the results of the Tables~\ref{table_hv} and ~\ref{table_igd}, we discuss the results of every priority function. The "spectral" norm as priority function lead to several good results in median of the HV values while had very good results in the median of the IGD values. We highlight that the result of the "spectral" norm in the function UF-5 were clearly better than the other results. The relative improvement function was first introduced in the context of the unconstrained MOEA competition in the CEC 2009~\cite{zhang2009performance}, being the winner of that competition~\cite{zhang2008multiobjective}. This competition introduced the UF benchmark functions, so it came to us with no surprising the good results from the relative improvement priority function. In terms of HV values, in all but in the UF-3 benchmark function it had the best result, being significantly different from the MOEA/D-DE in every single case. The results of the relative improvement in terms of median values of the IGD values was not as remarkable, but still very good.



 From the both the IGD- HV- evolution graph it has a strange regressive peak, while it seems like none of the variants had converged.

\subsection{DTLZ benchmark functions}



\subsection{Lunar Landing Problem}


Figure~\ref{HVS} (a) show box-plot the results found in the Lunar Landing  benchmark problem in terms of the HV values. Both priority functions related to diversity, MRDL and "spectral" norm, improved a lot the performance of the MOEA/D-DE. On the other hand, the relative improvement lead to a diminish of the standard deviation, with just a slight improvement on the performance.  Figure~\ref{evolution_hv} (a) show how the values of the HV evolve over the iteractions of the algorithms. All of the variants had converged around 20000 evaluations, with the exception of the MRDL (it converged with 40000), using around half of the total number evaluations (70000) to converge.

Changing to the results of the Table~\ref{table_hv}, we discuss the results of the priority functions.  We highlight that the result of "spectral" norm since it was clearly better than the other priority function results in terms of median of HV values as it is statistically different from MOEA/D-DE. Although the results of MRDL were not as good as the previous one, it is still has impressive median HV values results, but without statistically significant difference from MOEA/D-DE. The relative improvement improved the results of MOEA/D-DE but not as much as in the case of the UF benchmark functions, where it had frequently the best results. Here, on the other hand, the main improvement was in terms of diminishing the standard deviation.

%Since both measure diversity (in the decision space or in the objective space) we understand that focusing on this characteristics or similar might be the reason for that. 
%
%The best algorithm is the one that uses Norm, 0.93HV. It is the only one that is stats different from None, although by median aLl priority functions are better. For all , we have high sd. All but MRDL had lower sd than None. The median of None, RI and random are around around 0.7 HV, while the MRDL gets to 0.86 and norm is 0.93.
